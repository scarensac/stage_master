\documentclass{beamer}
\usepackage[english]{babel}
\usepackage[utf8]{inputenc}
\usepackage{times}
\usepackage[T1]{fontenc}
\usepackage{mathabx}
\usepackage{amsmath}
\usepackage{extarrows}
\usepackage{cancel}
\usefonttheme{professionalfonts}
\usepackage{cutwin}
\usepackage{lipsum}

\usepackage[nogradient, nobackground]{beamerthemeliris} % options: nogradient,nobackground

%\graphicspath{ {presentation_homotopic_thinning/} }

\setbeamercovered{transparent=5}

\title[Geodesic Distance]{Geodesic Distance and Metrics on Digital Surface}
%\subtitle{}

\author
[Thomas Caissard - thomas.caissard@etu.univ-lyon1.fr]
{T. Caissard }%\inst{1}}

\institute%[XXX]
{
	%\inst{1}%
	{\bf Laboratoire d'InfoRmatique en Image et Syst\`emes d'information} \\
	{ \scriptsize{
			LIRIS UMR 5205 CNRS/INSA de Lyon/Universit\'e Claude Bernard Lyon 1/Universit\'e Lumi\`ere Lyon 2/Ecole Centrale de Lyon\\
			INSA de Lyon, b\^atiment J. Verne\\
			20, Avenue Albert Einstein - 69622 Villeurbanne cedex\\
			\url{http://liris.cnrs.fr}}
	}
}

\AtBeginSubsection[]
{
	\begin{frame}<beamer>
		\frametitle{Plan}
		\tableofcontents[currentsubsection]
	\end{frame}
}

\AtBeginSection[]
{
	\begin{frame}<beamer>
		\frametitle{Plan}
		\tableofcontents[currentsection]
	\end{frame}
}

\usepackage{tcolorbox}
\tcbuselibrary{theorems}
\tcbuselibrary{skins}

\newcommand{\gatherblock}[2][]{\begin{gather*}\tcboxmath[#1]{#2}\end{gather*}}


\begin{document}

\begin{frame}
	\titlepage
\end{frame}

\begin{frame}	
	\frametitle{Introduction}
	\begin{block}{Objectives}
		\begin{itemize}
			\item Implement, test and compare two geodesic distance algorithms.
			\item Inject metric properties inside Discrete Exterior Calculus and test convergence.
		\end{itemize}
	\end{block}
\end{frame} 

\section{Geodesic Distance and DEC}

\begin{frame}
	\frametitle{What is a geodesic ?}
	
	\begin{block}{Geodesic Distance}
		$\gamma : \text{[a,b]} \rightarrow M$ a curve on a manifold $M$ with length:
		\gatherblock{L(\gamma)=\int_a^b \sqrt{g_{\gamma(t)} (\dot{\gamma}(t),\dot{\gamma}(t))dt },}
		
		$g$ is a metric tensor, $\dot{\gamma}$ is the derivative of $\gamma$.
		
		The geodesic distance between $P$ and $Q$ is the infinimum length between all the $\gamma$.
			
	\end{block}
	
\end{frame} 

\begin{frame}
	\frametitle{What is a geodesic ?}

	\begin{figure}
		\includegraphics[width=0.25\textwidth]{images/greatCircle.png}
	\end{figure}
	\begin{center}
	
	\gatherblock[colback=red!30]{\text{Geodesic is not unique !}}

	\end{center}
\end{frame}

\subsection{Geodesic in Heat (GH)}

\begin{frame}	
	\begin{itemize}
		\item Geodesics in heat: a new approach to computing distance based on
		heat flow \cite{DBLP:journals/tog/CraneWW13}
		\item Distance maps with heat equation
		\item Solving variational problems
	\end{itemize}
	
	\begin{figure}
		\includegraphics[width=0.5\textwidth]{images/rabbit.png}
		\vspace{-25px}
		\caption{\cite{DBLP:journals/tog/CraneWW13}}
	\end{figure}
\end{frame} 

\begin{frame}
	\frametitle{The algorithm}
	
	\begin{block}{Algorithm}
		\begin{itemize}
			\item Integrate $u_t=\Delta u_0$ for some fixed $t$. $u_t : M\times\mathbb{R}\rightarrow\mathbb{R}$: time dependent time propagation function.
			\item Evaluate the vector field $X = \displaystyle - \frac{\nabla u_t}{||\nabla u_t||}$.
			\item Solve the Poisson equation $\Delta\phi = \nabla \cdot X$.
		\end{itemize}
	\end{block}
	
	\begin{figure}[ht!]
		\centering
		\includegraphics[width=0.8\textwidth]{images/GH_graph.jpg}
		\caption{\cite{DBLP:journals/tog/CraneWW13} }
	\end{figure}
\end{frame}

\subsection{Earth Mover's Distance (EMD)}

\begin{frame}	
	\begin{itemize}
		\item Earth mover's distances on discrete surfaces \cite{DBLP:journals/tog/SolomonRGB14}
		\item Probabilist measures displacment using Optimal Transport theory
		\item Solving variational problems
	\end{itemize}
	
		\begin{figure}
			\includegraphics[width=\textwidth]{images/horse.png}
			\caption{\cite{DBLP:journals/tog/SolomonRGB14}}
		\end{figure}
\end{frame} 

\begin{frame}
	\frametitle{Optimal Transport}
	Geodesic distance:
	\gatherblock{
	\mathcal{W}(\mu_0,\mu_1) \equiv \underset{\pi\in\Pi(\mu_0,\mu_1)}{\text{inf}}\iint_{M\times M}d(x,y)d\pi(x,y)}
	where $\mu_0$ and $\mu_1$ are probabilistic measures. $d(\cdot,\cdot)$ the distance.
	\begin{figure}
		\includegraphics[width=\textwidth]{images/transport.png}
	\end{figure}
\end{frame}



\subsection{Discrete Exterior Calculus (DEC)}

\begin{frame}	
	
\end{frame} 

\section{Geodesic Distance on Digital Surface}

\begin{frame}	
	
\end{frame} 

\subsection{GH versus EMD}

\begin{frame}	
	
\end{frame} 

\subsection{Metric embedding}

\begin{frame}	
	
\end{frame} 

\section{Conclusion}

\begin{frame}	
	
\end{frame} 

\begin{frame}[allowframebreaks]
	\frametitle{References}
	\bibliographystyle{amsalpha}
	\bibliography{biblio.bib}
\end{frame}
	
\end{document}
